\frontmatter

%! Auswahl der Titelseite
%Titelseite

\begin{titlepage}
\begin{center}

\textbf{\Huge Praxisbeleg}\\
\vspace{1.5cm}
\LARGE{\titel \\}
\vspace{1.5cm}
\end{center}
\begin{flushleft}
\large{
\begin{tabular}{l l r}
\vspace{1.0cm}
\textbf{Vorgelegt am:}\quad\quad\quad & \abgabedatum\\

\textbf{Von:}           ~ & \textbf{\autor1}\\

\textbf{Studiengang:}   ~ & \studiengang \\
\vspace{1.0cm}
\textbf{Studienrichtung:} ~ & \studienrichtung \\
\vspace{1.0cm}
\textbf{Seminargruppe:} ~ & \seminargruppe \\

\textbf{Matrikelnummer:} ~ & \matnumeins \\

\textbf{Gutachter:}     ~ & \betreuereins \\ ~ & (\institutioneins)\\
                        ~ & \betreuerzwei \\ ~ & (\institutionzwei)\\
                        
\end{tabular}}
\end{flushleft}
\end{titlepage}
\newpage
%%Titelseite

\begin{titlepage}
\begin{center}

\textbf{\Huge Projektarbeit}\\
\vspace{1.5cm}
\LARGE{\titel \\}
\vspace{1.5cm}
\end{center}
\begin{flushleft}
\large{
\begin{tabular}{l l r}
\vspace{1.0cm}
\textbf{Vorgelegt am:}\quad\quad\quad & \abgabedatum\\

\textbf{Von:}           ~ & \textbf{\autoreins}\\
                        ~ & \textbf{\autorzwei}\\
\vspace{1.0cm}
                        ~ & \textbf{\autordrei}\\

\textbf{Studiengang:}   ~ & \studiengang \\
\vspace{1.0cm}
\textbf{Studienrichtung:} ~ & \studienrichtung \\
\vspace{1.0cm}
\textbf{Seminargruppe:} ~ & \seminargruppe \\

\textbf{Matrikelnummer:} ~ & \matnumeins \\
                         ~ & \matnumzwei \\
\vspace{1.0cm}
                         ~ & \matnumdrei \\
\textbf{Gutachter:}     ~ & \betreuereins \\ ~ & (\institutioneins)\\
                        ~ & \betreuerzwei \\ ~ & (\institutionzwei)\\

\end{tabular}}
\end{flushleft}
\end{titlepage}
\newpage


%! nicht benötigte Verzeichnisse hier auskommentieren
%? Inhaltsverzeichnis
\tableofcontents
\newpage

%? Abbildungsverzeichnis
\listoffigures
\newpage

%? Tabellenverzeichnis
\listoftables
\newpage

%? Programmcodeverzeichnis
\listofcodes
\newpage

%? Formelverzeichnis
\listofformeln
\newpage

%? Abkürzungsverzeichnis
\include{inhalt/Abkürzungen}


%! Hier beginnt der eigentliche Inhalt der Arbeit.
\mainmatter

%! Eine der folgenden Zeilen wieder aktivieren und entsprechende Datei ablegen
%? druckt Firmenlogo ab Kapitel 1 in Kopfzeile - SVG ist, auch bei kleinen Grafiken, zu bevorzugen, wenn vorhanden
%\lohead{\includesvg[height=8mm,inkscapelatex=false]{bilder/firmenlogo.svg}}
%\lohead{\includegraphics[height=8mm]{bilder/firmenlogo.png}}

%! Es ist möglich, die ganze Arbeit in eine Datei ("Kapitel1.tex") zu schreiben.
%! Sollten mehr Struktur und Übersicht gewünscht werden, können noch zusätzliche Dateien angelegt werden,
%! diese müssen nach dem selben Schema wie "Kapitel1.tex" hinzugefügt werden:
%Alle Seiten beginnen mit der Oberüberschrift
\section{Einleitung}
Die ist eine PDF mit der Vorlage erstellt, um die Funktion und Formatierung dieser zu zeigen.

Diese Vorlage\onlinezitat{SCZYRBA2020} ist ein Gemeinschaftsprojekt im Rahmen unseres Studiums.
Der Docker-Container\onlinezitat{HILLE2021} gehört dazu.
Der ganze Spaß hält sich an die \ac*{HAWA} der BA-Glauchau.

Weitere Hinweise befinden sich in der Readme.md oder im Wiki.
Eine ausführliche Dokumentation zu diesem Dokument wird folgen.

\bild[0.5]{ba-gc-logo}{Text zu einem Bild}{mein-label}

\section{Kapitel 1}
\blindtext

%? Quellen- und Literaturverzeichnis
\printbibliography

%? Anhang
%!	Anhang

\clearpage
\appendix
\clearpage

%! Section Befehl wird umgeschrieben, damit keine Überschriften mehr angezeigt werden
%!Kann falls Überschriften gewollt sind entfern werden oder erst später eingefügt
% Beginn 
\renewcommand{\section}[1]{
\par\refstepcounter{section}
\sectionmark{#1}
\addcontentsline{atoc}{section}{\protect\numberline{\thesection}#1}
\lohead{\textnormal{#1}}
} % Ende

%! Hier kann man sich anpassen, wie Abbildungen im Anhang dargestellt werden.
%! Bitte eins der beiden Auskommentieren
%? Möglichkeit 1 ohne Nummerierung und ohne Abbildung davor 
%\renewcommand{\bild}[3][1.0]{\begin{figure}[H]
%	\centering
%	\includegraphics[width=#1\columnwidth]{bilder/#2}
%	\caption*{#3}
%	\label{fig:#3}
%	\end{figure}}

%? Möglichkeit 2 mit Nummerierung und Abbildung, aber nicht im Abbildungsverzeichnis
\renewcommand{\bild}[3][1.0]{\begin{figure}[H]
	\centering
	\includegraphics[width=#1\columnwidth]{bilder/#2}
	\caption[]{#3}
	\label{fig:#3}
	\end{figure}}

%! Anhang 1
\section{Erster toller Anhang}
Hihi hier kommt eigentlich ein Anhangverzeichnis hin :D
\clearpage

%! Anhang 2
\section{Inhalt der CD}
CD mit folgenden Inhalten:
\begin{itemize}
	\item dieses Dokument
	\item Latex Dateien
	\item Youtube-Video als Bonus
\end{itemize}
 \clearpage

%! Eidestattliche Erklärung, hier zwischen Version für einen oder mehrere Autoren umschalten
\input{inhalt/Erklärung_Praxisbeleg}
%\input{inhalt/Erklärung_3Autoren}


