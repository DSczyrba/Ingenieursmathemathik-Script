

\documentclass[12pt, captions=nooneline, titlepage, footsepline, headsepline, toc=sectionentrywithdots]{scrartcl}
\usepackage[ngerman]{babel}
\usepackage{amsmath}
\usepackage{amssymb}
\usepackage{amsthm}
\usepackage{tabularx}
\usepackage{setspace} 
\usepackage{booktabs}
\usepackage{svg}
\usepackage{graphicx}
\usepackage{float}
\usepackage[a4paper,lmargin={2.5cm},rmargin={2.5cm},tmargin={2cm},bmargin={2cm}]{geometry}
\usepackage[backend=biber, style=alphabetic, citestyle=numeric]{biblatex}
\usepackage{csquotes}
\usepackage{helvet}
\usepackage[hidelinks]{hyperref}
\usepackage[printonlyused]{acronym}
\addbibresource{literatur.bib}

%Kopf- und Fußzeile
\usepackage[automark]{scrlayer-scrpage} 
\pagestyle{scrheadings} 
\clearscrheadings 
\clearscrplain 
\rohead{\headmark} 
\lofoot{} 
\cofoot{} 
\rofoot{\pagemark}
\makeatletter
\usepackage{geometry}
\geometry{a4paper,
          left=25mm,right=25mm,top=20mm,bottom=20mm,
          includehead=false, % Kopfzeile außerhalb des Textkörper, also im Rand
          includefoot=false,
          headheight = \baselineskip,
          headsep = \dimexpr\Gm@tmargin-\headheight-10mm,
          footskip = \dimexpr\Gm@bmargin-10mm,
          %showframe,
          bindingoffset=0mm}
% Kopfzeile 1,0 cm Abstand zum Blattrand
% Fußzeile 1,0 cm Abstand zum Blattrand
\makeatother
\samepage
\linespread{1.3}
\newcommand\frontmatter{%
    \cleardoublepage
  %\@mainmatterfalse
  \pagenumbering{Roman}}

\newcommand\mainmatter{%
    \cleardoublepage
 % \@mainmattertrue
  \pagenumbering{arabic}}

\newcommand\backmatter{%
  \if@openright
    \cleardoublepage
  \else
    \clearpage
  \fi
 % \@mainmatterfalse
   }

% ? Ab hier beginnen eigene Befehle um den Umgang zu erleichtern   

\newcommand{\absatz}{\vspace{12pt}\noindent}
\newcommand{\logisch}[1]{$``#1``$}
\newcommand{\bild}[3][1.0]{\begin{figure}[H]
                      \centering
                      \includegraphics[width=#1\columnwidth]{bilder/#2}
                      \caption{#3}
                      \label{fig:#3}
                      \end{figure}}